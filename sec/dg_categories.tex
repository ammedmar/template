% !TEX root = ../st_st.tex

\section{Differential graded categories as $\omega$-categories}

\subsection{Abelian $\omega$-categories and a Dold--Kan correspondence}

THE USUAL DOLD-KAN SEEMS TO NEED AN INVERSE DIFFERENT FROM THE NERVE. MAYBE THE NERVE HAS ITS OWN INVERSE THAT IS LIKE THE INVERSE OF THE $\mu$ (or $\lambda?$) functor of Steiner.

\SS Let $\GAb$ be the category of \defn{globular abelian groups}.
The \defn{nerve} $\nerve \colon \Ch \to \GAb$ is defined on chain complex by
\[
\nerve(C)_n = \Hom(\chains(\bG^n), C),
\]  
where the right hand side denotes chain maps.

\SS Let us consider the functor of \defn{globular chains} $\chains \colon \bG\Set \to \Ch$ sending a globular set $X$ to the based augmented chain complex with basis $X$ and boundary $\ta - \so$.
By a standard argument, this functor is completely determined by its restriction to the full subcategory of $n$-globes.
The \defn{globular nerve} $\nu \colon \bach \to \bG\Set$ is the right adjoint of $\chains$.
Explicitly, an $n$-cell of $\cells(C)$ is an augmentation-preserving non-negative chain map $\chains(\bG^n) \to C$.

\SS\label{def:Steiner-diagram}
We introduce a relation $<_k$ on basis elements of degree greater than $k$ for $k \in \N$ as follows:
$b <_k b'$ if and only if there is a basis element appearing with nonzero coefficient in both~$\angles{b}^+_k$ and~$\angles{b'}^-_k$.
The basis is said to be \defn{loop-free} if for each $k \geq 0$ the transitive closure of $<_k$ is antisymmetric, i.e., if it defines a partial order.
An augmented based chain complex is said to be a \defn{Steiner diagram} if its basis is both unital and loop-free.
The full subcategory of Steiner diagrams is denoted by $\wDiag$.
A satisfying property of the globular nerve of Steiner diagrams is that, if $\angles{c}$ is a characteristic map, then $c$ and all $\angles{c}_k^\epsilon$ are sums of distinct basis elements (\cite[Thm.~4.1]{steiner2012opetopes}).

In many cases, instead of loop-freeness is easier to check the following stronger condition.
We introduce a relation $<_\N$ on all basis elements declaring $b <_\N b'$ if and only if $b \in \bd^-(b')$ or $\bd^+(b) \ni b'$.
The basis is said to be \defn{strongly loop-free} if its transitive closure is antisymmetric.
As shown in \cite[Prop.~3.7]{steiner2004omega}, strongly loop-free bases are loop-free.
We refer to a Steiner diagram satisfying this condition as a \defn{strong Steiner diagram}.

\SS Extending maps by $0$ to define identities and using the additive structure of non-negative chain maps to define compositions, we endow the globular nerve $\cells(C)$ of a Steiner diagram with an $\omega$-category structure.
Explicitly, given an $n$-cell $f$ of $\cells(C)$, we abuse notation and define $f \colon \chains(\bG^{n+1}) \to C$ to be the only morphism with $f(e_{n+1}) = 0$ making
\[
\begin{tikzcd}
	\chains(\bG^{n+1}) \arrow[dashed, r, "f"] & C \\
	\chains(\bG^n) \arrow[u, shift left=4pt, "\sigma_*"] \arrow[u, shift right=4pt, "\tau_*"'] \arrow[ur, bend right, "f"'] &
\end{tikzcd} 
\]
commute.
To define compositions, consider the following commuting diagram
\[
\begin{tikzcd}
	\chains(\bG^{k}) \rar["\sigma_*"] \dar["\tau_*"'] \arrow[dr, "h"]& \chains(\bG^{n}) \dar["f"] \\
	\chains(\bG^n) \arrow[r, "g"'] & C.
\end{tikzcd} 
\]
Then, the $k$-composition of $f$ and $g$ is given by $f \circ_k g = f + g - h$. 

\medskip

As stated earlier, the functor $\cells \colon \wDiag \to \wCat$ is full and faithful, and the image of a Steiner diagram is a category freely generated by its atoms (\cite[Thm~5.11]{steiner2004omega}).
This makes the category of Steiner diagrams into a well-suited model of pasting diagrams.

\subsection{Differential graded categories as $\omega$-categories}

s